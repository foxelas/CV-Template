%!TEX TS-program = xelatex
\documentclass[]{friggeri-cv}

\usepackage{comment}

\begin{document}
\header{eleni}{aloupogianni}
       {information engineer}

% Need to compile 2 times for the top bar to appear 

% In the aside, each new line forces a line break
\begin{aside}
  %\includegraphics[scale=0.5]{elena2.jpg}
  \section{about}
   Call me Elena
   Nationality: Greece
   Gender: Female
  \section{contact}
    %〒158-0095
	%東京都世田谷区瀬田
	Setagaya Ward, Tokyo 
	158-0095, Japan
    \href{mailto:work.jpn@outlook.com}{work.jpn@outlook.com}
    \href{https://foxelas.github.io}{foxelas.github.io}
    \section{programming}
    MATLAB, Python %{\color{red} $\varheartsuit$}
    C\#, R, SQL, VBA
    CSS, HTML
    Bash, Asm
    \section{tools}
    \LaTeX, Git
    KiCad, OrCad 
    Relux, Weka, Praat
    Webots, AVR studio
    \section{os}
	Windows
	Ubuntu Linux 
	Android, iOs 
    \section{skills}
    problem-solving
    %task management
    teamwork, organising
    communication skills 
    positive attitude
    perseverance
    %creativity, . Ability to work under pressure, leadership     
    \section{interests}
    medical imaging
    multi-spectral analysis
    color analysis, signals
    pattern recognition 
    %languages, education
\end{aside}

\section{education}

\begin{entrylist}
  \entry
    {since 2019}
    {Ph.D.}
    {Tokyo Institute of Science and Technology, Japan}
    {in Information and Communications Engineering\\
    Human-Centered Science and Biomedical Engineering Course\\
    Funded by Japanese Government (Monbukagakusho: MEXT) Scholarship\\
    Member of Obi laboratory \\
    Specialized in: Multi-Spectral Image Analysis for Medical Purposes.}
  \entry
    {2017-2019}
    {M.Eng.}
    {Tokyo Institute of Science and Technology, Japan}
    {in Information and Communications Engineering (GPA 4.16/4.50)\\
    Funded by Japanese Government (Monbukagakusho: MEXT) Scholarship\\
    Member of Ohyama-Obi laboratory \\
    Thesis \emph{``Skin Cancer Detection Using Multi-Spectral Macropathology Images''}, under the supervision of Associate Professor Takashi Obi}
  \entry
    {2017-2017}
    {Research student}
    {Tokyo Institute of Science and Technology, Japan}
    {Intensive Japanese language course and preparation for graduate studies}
  \entry
    {2011-2016}
    {5-Year Joint B.Sc. and M.Sc.}
    {National Technical University of Athens, Greece}
    {in Electrical and Computer Engineering (Grade 8.26/10)\\
    Supported by Eurobank Scholarship (1000€) for top-ranking entrance score \\
    Specialized in: Signals, Computer Systems, Electronics, Bioengineering \\
    Diploma Thesis \emph{``Application of the   Curvelet Transform on Ultrasound Images of the Carotid Artery; Curvelet-based Feature Extraction and their Physical Meaning''}, under the supervision of Professor Konstantina Nikita}
%  \entry
%    {2008-2011}
%    {High-School Diploma with Honors}
%    {Geniko Lykeio Astrous, Arcadia, Greece}
%    {Scored 19.8/20 in nationwide university entrance examination.}
\end{entrylist}

\section{experience}
\begin{entrylist}
  \entry
    {since 2020}
    {Teaching Assistant}
    {Saitama Medical University, Hidaka, Japan}
    {Guidance and assistance in basic electronics experiments.}
  \entry
    {since 2019}
    {Teaching Assistant}
    {Tokyo Institute of Technology, Ota, Japan}
    {Problem identification, group-work management and progress evaluation.}
  \entry
  	{Feb-Mar2019}
  	{Research Assistant}
  	{Tokyo Institute of Technology, Ota, Japan}
  	{Inter-university project on ``Assisting skin cancer macropathological diagnosis using multi-spectral image'' in collaboration with Saitama Medical University, Chiba University and Olympus Corporation.}
  \entry
    {since 2017}
    {Software Developer (Part-time)}
    {IDAY Yamazaki Johosekei, Chiyoda, Japan} % 山崎情報設計
    {Data analysis, modelling and support of financial risk management software.}
\begin{comment}  
  \entry
    {since 2017}
    {Private lessons}
    {Part-time Job}
    {\emph{Tutoring Greek, English and Mathematics to elementary grade students.}}
  \entry
    {2018-2019}
    {Global Education Project(GEP)}
    {Occasional Part-time Job}
    {\emph{Facilitating English learning and cultural exchange among Japanese high school students in 1-day programs.}}
  \entry
    {08/2017}
    {English Summer Camp in Nasu}
    {Summer job.}
    {\emph{Camp captain. Facilitating English learning and cultural exchange among Japanese middle school students in a 3-day program.}}
  \entry
    {2007-2016}
    {Municipal Community of Agios Ioannis, Greece}
    {Summer volunteering.}
    {\emph{Secretary \& supervisor in annual children athletic games.}}
\end{comment}
\end{entrylist}

\section{publications}
\begin{entrylist}
  \entry
    {Journal}
    {Eleni Aloupogianni, Hiroyuki Suzuki, Takaya Ichimura et al }
    {2019}
    {\emph{``Binary Malignancy Classification of Skin Tissue using Reflectance and Texture Features from Macropathology Multi-Spectral Images''}, IIEEJ Transactions on Image Electronics and Visual Computing, Vol. 7, No. 2, Dec. 2019.}
	
	\entry 
	{More} 
	{Full list available on}
	{}
	{\url{https://foxelas.github.io/publications.html}}
	
%\entry
% {Conference}
%  {Eleni Aloupogianni, Hiroyuki Suzuki, Takaya Ichimura et al }
%   {2019}
%    {\emph{``Binary Malignancy Classification of Skin Tissue using Reflectance Features from Macropathology Multi-Spectral Images''}, In: JAMIT Annual Meeting 2019, Nara Japan, 24-26 Jul. 2019, pp. 645-647.}
    
\end{entrylist}
 


     
\section{languages} % \& skills
%\vspace{.5cm}
\begin{minipage}{0.5\textwidth}
\begin{tabular}{ll}
Greek: & \grade{5}   Native\\
English:& \grade{4.5} Full working proficiency \\
Japanese: & \grade{4} JLPT N2\\
German: & \grade{3}  Goethe-Zertifikat B2\\
%French: & \grade{1}  \\ %,Spanish
\end{tabular}
\end{minipage}

\begin{comment}

\begin{minipage}{0.5\textwidth}
\begin{center}
\begin{tikzpicture}[overlay,thick,scale=0.6, every node/.style={scale=0.75}]
\foreach \i [evaluate={\j=\i+1;}] in {0,...,\nrows}{
  \pgfplotstablegetelem{\i}{language}\of{\data}\let\language=\pgfplotsretval
  \pgfplotstablegetelem{\i}{years}\of{\data}\let\years=\pgfplotsretval
  \pgfmathsetmacro\years{\years/2}
  \path [sector=\i] (\i*\step:1) (\i*\step:1+\years) 
    arc (\i*\step:\j*\step:1+\years) -- (\j*\step:1)
    arc (\j*\step:\i*\step:1) -- cycle;
  \pgfmathparse{int(\years>2)}
  \ifnum\pgfmathresult=1
    \node [text=white, font=\bfseries] 
      at (\i*\step+\step/2:1+\years/2) {\language};
  \else
    \node [text=lightgray, font=\bfseries]
      at (\i*\step+\step/2:1+\years+1/2) {\language};
  \fi
}
\end{tikzpicture}
\end{center}
\end{minipage}
\end{comment}

\end{document}
